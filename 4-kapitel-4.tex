\chapter{Kapitel 4}
\label{cha:kapitel-4}

\section{Behandlungsansätze}
\label{sec:behandlungsansätze}
Prokrastination ist bisher noch nicht in die gängigen Klassifikationssysteme psychischer Störungen (DSM-5, ICD 10) aufgenommen worden. Dementsprechend gibt es bisher kaum systematisch evaluierte Behandlungsansätze, die auf die Behandlung einer isolierten Aufschiebe-Symptomatik abzielen. Ratgeber (z. B. Rückert, 2011\cite{wiki},) liefern in der Regel allgemeine Tipps zu effizienterem Arbeiten, deren Wirksamkeit aber nicht wissenschaftlich belegt wurde. Ziel einer Behandlung sollte eine Verbesserung der Selbststeuerung sein.

Ein aktueller, manualisierter Ansatz\cite{hoecker:2013} aus der kognitiven Verhaltenstherapie besteht aus verschiedenen Bausteinen, deren Wirksamkeit in verschiedenen Studien\cite{wiki} belegt wurde und beachtliche Effektstärken aufwies. Innerhalb eines Modells zur Realisierung von Absichten (sog. Rubikonmodell) können die kritischen Punkte im Ablauf identifiziert werden. Diese bestehen oft in der Planungsphase und Handlungsvorbereitung sowie im Übergang zur Ausführung, wo insbesondere die Handlungsinitiierung relevant ist. Die Module (Pünktlich Beginnen, Realistisch Planen, Arbeitszeitrestriktion und Bedingungsmanagement) können sich einzeln oder in Kombination an eine Selbstbeobachtung mithilfe eines Arbeitstagebuchs anschließen. Während die beiden ersten Module die kritischen Phasen des Rubikonmodells betreffen, kann ein drittes Modul zum Einsatz kommen, das aus Arbeitszeitrestriktion und Bedingungsmanagement besteht. Im Modul „Pünktlich Beginnen“ geht es darum, einen vorgenommenen Zeitpunkt wirklich einzuhalten, konkrete Vorkehrungen dafür zu treffen und sich aktiv in die richtige Arbeitsstimmung zu bringen. Das Modul „Realistisch Planen“ arbeitet mit konkreten Gelegenheitsvorsätzen bezüglich Zeit, Ort, geplantem Inhalt und Dauer sowie einem Umfang der geplanten Aufgabe, der dem Leistungsvermögen angepasst ist, sowie motivierenden Gedanken. Das Modul „Arbeitszeitrestriktion“ ist die modernste und derzeit erfolgreichste Methode zur Behandlung von Prokrastination – hier wird mittels Arbeitszeitverknappung die Attraktivität der Arbeitszeit und damit der Aufgabe erhöht, da zu Beginn lediglich zwei feste Zeitfenster pro Arbeitstag festgelegt werden, über deren Dauer hinaus nicht gearbeitet werden darf. Diese werden individuell an die zuvor untersuchte durchschnittliche Arbeitszeit angepasst und dürfen erst dann vergrößert werden, wenn die zuvor definierten Zeiten effizient genutzt wurden. Die Methode erhöht die tägliche Arbeitszeit, reduziert das Aufschieben, erhöht die Effizienz der Arbeitszeit und führt zu einer besseren Trennung von Arbeit und Freizeit.