\chapter{Zusammenfassung und Ausblick}
\label{cha:zusammenfassung}
		
In \Cref{cha:einleitung} wurde das Interesse an diesem spannenden Thema geweckt. Wir haben zusätzlich Informationen eingefügt die mit der eigentlichen Arbeit nichts zu tun haben, aber in der Aufgabenstellung von uns verlangt wurden. Diese dürfen vom Leser ignoriert werden, wobei der oder die Korrektor/in darauf hingewiesen sei, dass wir die geforderten Dinge umgesetzt haben, auch wenn sie nicht zum Thema passen.

In \Cref{sec:störungsbild} wird das Krankheitsbild näher erläutert und beschrieben mit welchen folgen Betroffene nach dem Studium zu rechnen haben. Mit erschrecken haben wir, in einer fiktiven, von uns durchgeführten Studie festgestellt, dass viele Studenten darunter leiden.

In \Cref{sec:diagnostik} beschreiben wir Faktoren die dazu beitragen krankheitsfördernd zu sein. Es gibt zwei Arten die unterschieden werden: das aktive Prokrastinieren und das passive Prokrastinieren. durch Fragebögen und persönlicher Einschätzung kann entschieden werden zu welcher Sorte der Patient gehört.

In \Cref{sec:behandlungsansätze} erklären wir mögliche Behandlungsansätze, wobei die Behandlung für jeden individuell abgestimmt sein sollte. Da die Prokrastination bisher noch nicht in die gängigen Klassifikationssysteme psychischer
Störungen aufgenommen worden ist, gibt es auch noch kaum systematisch evaluierte Behandlungsansätze. 


