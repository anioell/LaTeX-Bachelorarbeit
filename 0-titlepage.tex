%---------------------------------------------------------------------------
% Frontpage
%---------------------------------------------------------------------------

% Die Richtline zum Aufbau des Deckblatts von Bachelor- und Masterarbeiten
% findet sich hier:
% @see: http://www.uni-luebeck.de/fileadmin/uzl_ssc/PDF-Dateien/Richtlinie-Deckblatt-MINT-Abschlussarbeit-2012-10-18.pdf

\newcommand{\titlepageskip}{\vskip 20pt}

% @see: http://tex.stackexchange.com/questions/31705/different-margins-for-title-page
\newgeometry{top=1in,bottom=1in,right=1in,left=1.2in}
\begin{titlepage}

\title{Prokrastination}
\author{Anika Oellerich}

{\Large
	% 1. Offizielles Logo des Instituts, an dem die Arbeit angesiedelt ist. (Das offizielle Logo
	% enthält das Siegel der Universität zusammen mit dem Text "Universität zu Lübeck"
	% und darunter den Namen des Instituts.) Dieses Logo ist bei den Instituten zu
	% bekommen. Das Logo muss oben links platziert werden.
	\includegraphics[width=80mm]{Logo_Inst_Telematik_cropped}
	\vskip 44pt

	% 2. Optional: Noch einmal Name des Instituts und Angabe der Direktorin oder des
	% Direktors des Instituts.

	% 3. Titel der Arbeit in deutscher Sprache und ebenfalls in englischer Sprache. Dabei soll
	% die Sprache, in der die Arbeit verfasst wurde, als erste angeführt werden; die andere
	% Sprache kann weniger prominent dargestellt werden.
	% Auch bei englischsprachigen Studiengängen sollen die Titelblätter auf Deutsch sein.
	{\LARGE\bf Prokrastination\par}
	{\LARGE Procrastination\par}

	\titlepageskip
	% 4. Der Text "Bachelorarbeit" oder "Masterarbeit" (nicht "Bachelor-Arbeit" oder "Master-Arbeit").
	{\bf Bachelorarbeit}
	%{\bf Masterarbeit}

	\titlepageskip
	%5. Der Text "im Rahmen des Studiengangs"
	im Rahmen des Moduls\\
	%6. Der ausgeschriebene Name des Studiengangs (also beispielsweise "Informatik"
	%oder "Molecular Life Science", hingegen nicht "Bioinformatik" oder "MLS")
	{\bf Werkzeuge für das wissenschaftliche Arbeiten – CS2450}\\
	%7. Der Text "der Universität zu Lübeck"
	der Universit"at zu L"ubeck

	\titlepageskip
	%8. Der Text "Vorgelegt von" und der Name der Studentin oder des Studenten
	vorgelegt von\\
	{\bf Maurice Grefe}\\
	{\bf Anika Oellerich}

	\titlepageskip
	%9. Der Text "Ausgegeben und betreut von"
	ausgegeben und betreut von\\
	%10. Der Name der ersten Prüferin oder des ersten Prüfers. Dies ist immer gleichzeitig
	%die Betreuerin oder der Betreuer im Sinne der Prüfungsordnung.
	{\bf Prof.~Dr.~Max~Mustermann}

	% Diesen Teil entfernen, wenn die Arbeit KEINEN Unterstützer hatte
	\titlepageskip
	{
		%11. Optional der Text "Mit Unterstützung von" und der Name von weiteren Personen,
		%die die Betreuung besonders unterstützt haben. Beispielsweise können dies
		%wissenschaftliche Mitarbeiter sein oder Mitarbeiter von Firmen, wenn die Arbeit
		%extern geschrieben wurde.
		mit Unterstützung von\\
		{\bf Prof.~Dr.~X~Musterfrau}\\
		{\bf Dipl.-Inf.~Y~Mustermann}\\
	}

	% Diesen Teil entfernen, wenn die Arbeit NICHT in einer Firma entstanden ist
	\titlepageskip
	{
		%12. Optional ein Hinweis, dass die Arbeit zum Teil bei einer Firma entstanden ist wie
		%"Die Bachelorarbeit ist im Rahmen von Arbeiten bei Firma XY entstanden".
		%Die Arbeit ist im Rahmen einer Tätigkeit bei der Firma Muster GmbH entstanden.
	}

	\vfill 
	%13. Der Text "Lübeck, den" und das Abgabedatum.
	{
		Lübeck, den \today
	}

	% Diesen Teil entfernen, wenn "Im Focus das Leben" nicht drauf stehen soll
	%14. Optional der Text "Im Focus das Leben".
	{
		\titlepageskip
		Im Focus das Leben
	}
}
\end{titlepage}
\restoregeometry

\cleardoublepage

% Erklaerung
\begin{comment}
\newpage
\vspace*{7cm}
\centerline{\bf Erkl"arung}

\vspace*{1cm}
Ich versichere an Eides statt, die vorliegende Arbeit selbstst"andig und nur unter Benutzung
der angegebenen Hilfsmittel angefertigt zu haben.

\vspace*{3cm}
Lübeck, den \today 

\pagestyle{headings}

\cleardoublepage
\end{comment}
% Kurzfassung und Abstract

\section*{Kurzfassung}

Prokrastination (lateinisch procrastinare „vertagen“; Zusammensetzung aus pro „für“ und cras „morgen“),
auch extremes Aufschieben, ist eine Arbeitsstörung, die durch ein nicht nötiges Vertagen des Arbeitsbeginns 
oder auch durch sehr häufiges Unterbrechen des Arbeitens gekennzeichnet ist, sodass ein Fertigstellen der 
Aufgabe gar nicht oder nur unter enormem Druck zustande kommt. Dies geht fast immer mit einem beträchtlichen 
Leidensdruck einher. Pathologisches Aufschieben muss unterschieden werden vom alltäglichen Aufschieben bei
aversiven Aufgaben, das viele Menschen kennen (nur 1,5\% einer studentischen Population berichteten, gar 
nicht aufzuschieben), dem Vertagen von Aufgaben aufgrund anderer, nötiger Prioritätensetzung, sowie einem 
erfolgreichen Arbeiten kurz vor einer Frist, wodurch es weder zu Leistungseinbußen noch zu subjektivem Leiden 
kommt. Während umgangssprachlich häufig vom „Studentensyndrom“ gesprochen wird, handelt es sich bei 
Prokrastination um eine in der Gesamtpopulation vorkommende Arbeitsstörung, die besonders bei Personen 
zutage tritt, die hauptsächlich selbstgesteuert arbeiten müssen (z. B. Studenten, Anwälte, Journalisten,
Lehrer). Betroffene leiden meist dauerhaft darunter, berichten teilweise bereits zu Schulzeiten Probleme 
gehabt zu haben und erleben dies auch in ihrem späteren Berufs- und Privatleben.

Das Gegenteil der Prokrastination ist die Präkrastination\cite{kuenzel:2014}.

%
\vskip 3cm
%

\section*{Abstract}

Procrastination is the avoidance of doing a task that needs to be accomplished\cite{9781111987251}. It 
is the practice of doing more pleasurable things in place of less pleasurable ones, or carrying out less 
urgent tasks instead of more urgent ones, thus putting off impending tasks to a later time. Sometimes, 
procrastination takes place until the "last minute" before a deadline. Procrastination can take hold on
any aspect of life — putting off cleaning the stove, repairing a leaky roof, seeing a doctor or dentist,
submitting a job report or academic assignment or broaching a stressful issue with a partner. 
Procrastination can lead to feelings of guilt, inadequacy, depression and self-doubt.



\cleardoublepage

% Aufgabenstellung

\section*{Aufgabenstellung}

Suchen Sie sich einen interessanten und nicht zu langen Wikipedia-Artikel und verwandeln Sie diesen in 
eine von Ihnen verfasste fiktive Bachelorarbeit. Setzen Sie diese Arbeit mit LaTeX um. Ihre Arbeit muss 
dabei mindestens

\begin{itemize}
    \item ein Deckblatt,
    \item ein Inhaltsverzeichnis,
    \item eine deutsche und englische Kurzfassung,
    \item eine Einleitung, eine Zusammenfassung und zwei weitere Kapitel,
    \item zwei Abbildungen mit Bildunterschrift und
    \item ein Literaturverzeichnis
\end{itemize}

enthalten.

Das Literaturverzeichnis muss mindestens

\begin{itemize}
    \item ein Beitrag in einer Zeitschrift
    \item ein Buch und
    \item einen Beitrag im Berichtsband einer Konferenz
\end{itemize}

enthalten.

Erzeugen Sie ein Git-Repository, das den Quelltext und alle notwenigen Dateien zum Übersetzen des Quelltextes enthält. Das Repository darf weder die generierte PDF-Datei noch temporäre LaTeX-Dateien enthalten. Nutzen Sie dazu eine \texttt{.gitignore}-Datei. Veröffentlichen Sie dieses Repository auf Ihrem GitHub-Account. Bearbeiten Sie diese Aufgabe zu zweit. Geben Sie hier Ihre beiden Namen und den Link zum Repository an. (Ihre Namen müssen in der Arbeit selber nicht zwingend angegeben werden.)