\chapter{Kapitel 3}
\label{cha:kapitel-3}

\section{Bedingungen und Diagnostik}
\label{sec:diagnostik}
Das Verhalten tritt insbesondere dann zutage, wenn die Bedingungen zur Zielerreichung wenig konkret sind, aber auch, wenn die Aufgabe als besonders groß oder aus anderen Gründen als besonders aversiv wahrgenommen wird. In der Literatur wird ein behaviorales Bedingungsmodell zur Entstehung und Aufrechterhaltung des Aufschiebeverhaltens angenommen (vgl.\cite{hoecker:2013}). Danach hat man immer die Möglichkeit, eine Aufgabe später (Längskonkurrenz) zu erledigen oder jetzt (Querkonkurrenz) eine andere Tätigkeit zu wählen. Durch das Nicht-Ausführen der unangenehmen Tätigkeit kommt es einerseits durch den vorübergehenden Wegfall negativer Gefühle und Konsequenzen zu einer Spannungsreduktion und dadurch im Sinne von Konditionierungstheorien zu negativer Verstärkung, durch das Ausführen einer im Vergleich weniger unangenehmen, also positiveren Tätigkeit (z. B. Putzen) kommt es zudem zu positiver Verstärkung. Beides sorgt kurzfristig für ein besseres Gefühl, langfristig aber zu Leistungsrückstand, Stress, Selbstabwertung und schlechtem Gewissen. Dieses Modell sollte immer individuell angepasst werden – insbesondere wenn es um die individuellen prokrastinationsfördernden Bedingungen geht. Darauf weisen die Autoren ausdrücklich hin.

Damit dieses Verhalten als nichtselbstwertschädigend empfunden wird, kommt es bei Betroffenen oft zu einer Reihe von Rationalisierungen,[4] mit denen das Verhalten dann vermeintlich erklärt wird. Einige Forscher unterscheiden zudem zwischen aktivem und passivem Prokrastinieren:\cite{wiki}. Während der erste Typ (active/arousal) absichtlich bis zum letzten Moment wartet und dann aktiv durcharbeitet, zeigt der zweite (passive/avoidant) Vermeidung und Nichterledigen. Empirisch hängen beide aber stark miteinander zusammen \cite{wiki}, eine Unterscheidung scheint in der Praxis aufgrund des extrem hohen Zusammenhangs der beiden vermeintlichen Typen (Korrelation von r = 0.68) jedoch unangemessen\cite{wiki}. Je nach Kriterium und Studie bezeichnen sich 10 bis 75 \% der Befragten als „Aufschieber“, die geschätzte Häufigkeit hängt demnach extrem davon ab, wie eine Studie nach dem Aufschieben fragt\cite{wiki}. Prokrastination im Sinne „problematischen“ oder „extremen“ Aufschiebens ist deutlich seltener, Studien sprechen für eine Auftretenshäufigkeit von durchschnittlich 10 \%, auch nach den Kriterien (DKP) der Prokrastinationsambulanz der Universität Münster\cite{hoecker:2013}.

Zur Diagnostik sollten neben störungsspezifischen Instrumenten (Academic Procrastination State Inventory, APSI+\cite{wiki}; Allgemeiner Prokrastinationsfragebogen, APROF;\cite{hoecker:2013} DKP\cite{hoecker:2013}) auch Fragebögen zur Differentialdiagnostik eingesetzt werden, um nicht nur Komorbiditäten zu erfassen, sondern auch andere Erkrankungen zu ermitteln, die möglicherweise die Arbeitsstörung besser erklären. So könnte auch eine bestehende AD(H)S die Störung des Selbstmanagements erklären. Auch sollte abgeklärt werden, ob die Leistungseinbußen auf eine möglicherweise bestehende Depression zurückzuführen sind. Es sollte auch geprüft werden, ob eine zwanghafte oder narzisstische Persönlichkeitsstörung vorliegt, die die Durchführung oder das Beenden der Aufgabe behindern. Auch Prüfungsangst, Minder- oder Hochbegabung, kognitive Beeinträchtigungen, defizitäre Lern- und Arbeitstechniken oder Cannabis-Konsum können die Leistungsfähigkeit beeinträchtigen und sollten als Erklärungen ausgeschlossen werden.

\subsection{Grafik}

\myfig[0.75\textwidth]{10_Diagnosekreislauf_Paed_Diagn}{Diagnosekreislauf (eigene Darstellung)}
% \myfig{LuebeckHolstentor}{Das Holstentor in Lübeck}

Das Verfahren folgt einem Diagnosekreislauf siehe \Cref{fig:10_Diagnosekreislauf_Paed_Diagn} und ist ein sich wiederholender Prozess . 