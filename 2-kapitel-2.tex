\chapter{Kapitel 2}
\label{cha:kapitel-2}

\section{Störungsbild}
\label{sec:störungsbild}
Prokrastination bezeichnet ein Verhalten, das dadurch gekennzeichnet ist, dass Aufgaben trotz vorhandener Gelegenheiten und Fähigkeiten entweder nicht, oder erst nach sehr langer Zeit und dabei oft zu spät erledigt werden. Stattdessen werden häufig Alternativtätigkeiten ausgeführt, die relativ angenehmer sind und/oder unmittelbare Verstärkung ermöglichen (Bsp. Putzen). Es führt zu subjektivem Leiden, da die Betroffenen ihre Aufgaben gar nicht oder nur unter sehr großen Mühen fertigstellen.

In Analogie zu DSM-Kriterien anderer klinischer Störungen (Kriterien aus dem Diagnostic and Statistical Manual of Mental Disorders, die festlegen, ab wann jemand z.B. eine Depression oder eine bestimmte Angststörung hat) wurden Merkmale definiert, die mithilfe des DKP (Fragebogen zu den Diagnosekriterien Prokrastination \cite{hoecker:2013}) erfasst werden können\cite{download}.

Auch wenn häufig zwischen akademischer (= studentischer) und Alltagsprokrastination unterschieden wird, sind in der Regel beide Bereiche in ähnlicher Weise betroffen. Konsequenzen sind schlechtere Leistung und anhaltende Unzufriedenheit. Zusätzlich können die Betroffenen als Folge des Aufschiebens unter körperlichen und psychischen Beschwerden leiden (z. B. Muskelverspannungen, Schlafstörungen, Herz- und Kreislaufprobleme, Magen- und Verdauungsprobleme, innere Unruhe, Anspannung, Druckgefühl, Angst oder Hilflosigkeit). Zusammenfassend lässt sich sagen, dass die Betroffenen überwiegend nicht mehr das tun, was sie eigentlich tun wollen und häufig auch unter Selbstabwertung leiden. Prokrastination beeinträchtigt nicht nur das psychische Wohlbefinden, sondern kann zudem zu ernsthaften beruflichen und persönlichen Konsequenzen führen. Es ist nicht mit Faulheit zu erklären, sondern ist ein ernsthaftes Problem der Selbststeuerung.
